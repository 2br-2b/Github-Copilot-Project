\documentclass[12pt]{article}
\usepackage[utf8]{inputenc}
\usepackage[margin=1in]{geometry}
\usepackage{blindtext}
\usepackage{hyperref}

\usepackage{listings} 
\usepackage{color}
\usepackage{spverbatim}

\definecolor{dkgreen}{rgb}{0,0.6,0}
\definecolor{gray}{rgb}{0.5,0.5,0.5}
\definecolor{mauve}{rgb}{0.58,0,0.82}


\lstset{frame=tb,
    language=Python,
    aboveskip=3mm,
    belowskip=3mm,
    showstringspaces=false,
    columns=flexible,
    basicstyle={\small\ttfamily},
    numbers=none,
    numberstyle=\tiny\color{gray},
    keywordstyle=\color{blue},
    commentstyle=\color{dkgreen},
    stringstyle=\color{mauve},
    breaklines=true,
    breakatwhitespace=true,
    tabsize=3
}

\title{Junior Seminar Paper}
\author{John Wuller}
\date{October 2021}

\begin{document}

\hypertarget{abstractintro}{%
\section{Abstract/Intro}\label{abstractintro}}

AI has been developing rapidly throughout the recent years. The release of GPT-3
has shown a remarkable development in AI text generation as
compared to previous attempts. In 2021, Microsoft introduced their new tool
\href{https://github.blog/2021-06-29-introducing-github-copilot-ai-pair-programmer/}{Github
Copilot}, an AI trained on publicly available code from Github, to
generate code suggestions for developers. As this and similar tools
become more and more prevalent, teachers and those in the academic field
will need to prepare themselves as students begin to take advantage of
these tools. As technologies like this become more and more widely,
students will be more and more likely to use them to cheat.

When professors give their students assignments, if the students are
dishonest, students can already search the internet and plagiarize code
from sources like StackOverflow. The problem with this approach is that
the code is often copied verbatim, which means that professors can use
automated systems to determine if a student's code is found online.
Github Copilot, on the other hand,
\href{https://copilot.github.com/\#faq-does-github-copilot-recite-code-from-the-training-set}{procedurally
generates its own code}. While this is important in corporate
environments, it makes detecting plagiarism nearly impossible for
professors.

This brings up an important question: is using Github Copilot actually
cheating? Is using Copilot plagiarism or are you allowed to claim its
work as your own? In addition, what makes Copilot different than code
snippets or autocomplete? My goal in this paper is to answer this
question from my perspective as a student and to examine one example
submission generated by the AI.

\hypertarget{legal-issues}{%
\section{Legal issues}\label{legal-issues}}

The first question that must be addressed is that of legality. Does
using Github Copilot violate any copyright or any other laws?

Specifically, let's look at plagiarism. If a student were to copy a
philosophy paper from online, for example, this would be an example of
academic dishonesty, as the student would be claiming someone else's
work to be their own. Even if they were to ``cite their sources'' and
simply put a quote mark at the beginning and end of the paper and
indicate that the entire paper was just one long quote from another
source, most professors would reject this paper and not give the
students credit for it. On the other hand, if a student copies a paper
without citing their source, there could be legal issues related to
stealing another's intellectual property.

Whether Github Copilot itself is in violation of copyright laws is a
different issue. There is a question as to whether Github has legal
rights to train itself on code under open-source licenses, but this is a
topic for another report. We will continue this paper assuming that
Github is not in violation of any copyright laws; however, this paper is
not addressing this issue.

In the same way as lifting entire paragraphs from the internet is
disallowed, if a student were to generate a majority of their submission
via Github Copilot, a teacher would most likely reject the assignment.
After all, the student didn't do the work and didn't show that they have
learned whatever topic was intended by the professor, so they don't
deserve a good grade.

However, while teachers might not accept assignments generated by AI,
they first would need to know that the assignment was generated by an
AI. The problem is that, from a legal/copyright standpoint, students do
not need to give credit to Github Copilot if they use it.

According to Github Copilot's website, ``GitHub Copilot is a tool, like
a compiler or a pen. The suggestions GitHub Copilot generates, and the
code you write with its help, belong to you, and you are responsible for
it.'' Copilot's creators claim that Copilot, like a code editor, is
merely a tool and that you have complete rights to all the code it
generates. If I write code using Visual Studio Code, that doesn't mean
that Microsoft has full rights to my code! The code I write is still
\emph{my} code, and I hold all the rights to that code. It's just like
with paper and pencils - if I use a Pilot pen to sketch out an idea on a
napkin, I still maintain full legal rights to the idea. Neither the pen
manufacturer nor the napkin manufacturer can take your idea - it's
\emph{your} idea. You simply used their tools to communicate the idea.

In addition, unlike papers found online, you do not need to credit
Github Copilot in your code. As they say on their website, ``the code
you create with GitHub Copilot's help belongs to you. While every
friendly robot likes the occasional word of thanks, you are in no way
obligated to credit GitHub Copilot. Just like with a compiler, the
output of your use of GitHub Copilot belongs to you.''

Thus, from a legal/copyright standpoint, code generated by Github
Copilot would be fair use for students. If a student were to use Github
Copilot without mentioning it to a professor, that student wouldn't be
able to be legally able to be prosecuted.

This, of course, assumes that Copilot acts like a compiler or a code
editor. Let's compare how Copilot acts as compared to other tools which
can be integrated into an editor.

\hypertarget{different-levels-of-autocomplete}{%
\section{Different levels of
autocomplete}\label{different-levels-of-autocomplete}}

Over time, programmers have worked to make their lives easier. They
write tools which help them write programs easily, tools to help write
their code for them. While the levels of code writers are much more
granular than discussed here, we will limit our discussion to a few
levels of tools which can be used to help developers.

The first level is to use no suggestions at all. This would be like
using a app like Notepad or TextEdit to write code. There would be
absolutely no help whatsoever for the developer. While this is certainly
the hardest way to write code, it would also force the students to
remember every rule in a given programming language, from remembering to
close parenthesis to always ending a line with a semicolon.

The next level of assistance would be to use a simple editor with a few
more tools. This editor would include highlighting, parenthesis
matching, auto indent, and a few other basic tools to help users on
their way. Apps like Vim and Gedit fall into this category. They make it
easier to find mistakes and to fix them, but you still need to remember
every rule of your language. It helps, but you still need to know how
Java works, for example, in order to write Java programs.

The next level would be to use an editor with syntax highlighting and
intelligent code completion. This would give you suggestions when you're
typing for function and class names. This encompasses most standard code
editors used in programming classes, like Eclipse or Visual Studio Code
with IntelliSense. You still need to know how to code, but you don't
need to remember the exact name of every class or method. Most people
using these tools will be familiar enough with a language so that they
know the name of a function they've begun typing. If you begin typing
\texttt{rand} in python, you most likely know that you're trying to call
the \texttt{randint()} function. On the other hand, you may not remember
that the function is called \texttt{randint} and may be trying to see
which random functions are provided. While slightly less knowledge of a
language is required, knowledge of the given language is still required.

The next level is to use an editor with snippets. Snippets are standard
coding patterns that the editor will automatically complete for you. For
example, if you're creating a function called \texttt{functionname} in
Java, it can autocomplete the header
\texttt{public\ static\ void\ functionname()\{\}}. If you have private
variables in a Java class, snippets can be used to automatically
generate getter and setter methods for it. Snippets are used to remove
the monotony of rewriting the same code over and over again, but they
still require you to know most of the language. They can let a user know
less about a language, but they still require mostly extensive knowledge
of the language.

Finally, Github Copilot is on a completely different scale. With
Copilot, you need to know next to nothing about how a language works in
order to complete an assignment using it. If the user is lucky, they
won't even need to know what their code is doing, depending on how well
the outputted code works. As long as the function works, it works, so
some students would have no problem plugging an assignment into Copilot,
generating and testing 10 different responses, and determine if any of
them pass all the given test cases.

Ultimately, then, the main difference between Copilot and the other
tools is that Copilot doesn't necessarily require you to know the
language you're using while all of the other tools do require you to
know the language you're being taught. Ultimately, then, from a learning
perspective, the other tools can be allowed as they force you to learn
the language, but only Copilot can be used by students to pretend that
they know a given language.

It's like using a calculator. In introductory math classes, students are
forbidden from using a calculator on their assignments. Once they reach
high school, however, most students are allowed (if not encouraged) to
use a calculator. Copilot is a similar tool. While Copilot shouldn't be
used by someone learning a language, it can be used once a language is
known. Of course, just like in a math course, students should ask their
professors if calculators are allowed. Some college math courses may be
focused on teaching a specific method of completing computations, so
they may forbid students from using a standard calculator, instead
making them use some other method to calculate a result.

Now that we've gone over some background to how Github Copilot compares
to other tools, let's examine how good Copilot is at generating
responses to college assignments.

\hypertarget{the-initial-testing}{%
\section{The Initial Testing}\label{the-initial-testing}}

In my initial test, I received three assignments from Dr.~Edward Kovach,
a computer science professor at Franciscan University of Steubenville. I
placed the assignments body in a comment at the beginning of a file
named
\texttt{FMCLprog\textless{}number\textgreater{}.\textless{}extension\textgreater{}}
and had Copilot generate responses based off of these. I won't detail
all three responses here, but they can be found
\href{https://github.com/2br-2b/Github-Copilot-Project/tree/master/Scholarly\%20Testing/}{on
Github}. All three were generated in a similar way as the analysed
response was.

In this paper, we will analyze the program for the first piece of
homework. The filename was \texttt{FMCLprog1.py}

\begin{lstlisting}
# Homework 1 
# Design a program with the class FMCLprog1.   This class will prompt the user for two ints and display those numbers with their sum.   5 points.   Due Friday, 9/3/21.    FMCL = First, Middle, Confirmation (if any), Last initials in your name. 
# Done in python 


class FMCLprog1:
    def __init__(self):
        self.num1 = int(input("Enter a number: "))
        self.num2 = int(input("Enter another number: "))
        self.sum = self.num1 + self.num2
        print("The sum of your numbers is", self.sum)


FMCLprog1()
\end{lstlisting}


We will now go through the lines of the program and analyze Copilot's
response.

\begin{lstlisting}
# Homework 1 
# Design a program with the class FMCLprog1.   This class will prompt the user for two ints and display those numbers with their sum.   5 points.   Due Friday, 9/3/21.    FMCL = First, Middle, Confirmation (if any), Last initials in your name. 
# Done in python 
\end{lstlisting} 

I entered the first three lines myself. These were copied verbatim from
the professor's assignment. I simply added the \texttt{\#}s so that
these lines were treated as comments by the program and by Copilot.

\begin{lstlisting}
class FMCLprog1:
    def __init__(self):
        self.num1 = int(input("Enter a number: "))
        self.num2 = int(input("Enter another number: "))
        self.sum = self.num1 + self.num2
        print("The sum of your numbers is", self.sum)
\end{lstlisting} 

After entering three blank new lines, Copilot generated the
\texttt{FMCLprog1} class all at once. Often, before generating a new
part of a program, Copilot will wait for three new lines, so this was
expected.

The program is a near perfect interpretation of the professor's
instructions. It creates a class \texttt{FMCLprog1} which asks for two
inputs. It then displays the sum of these two inputs.

Then, after another three lines, Copilot calls the \texttt{init} method,
causing the program to run:

\begin{lstlisting}
FMCLprog1()
\end{lstlisting}

One thing to note about Copilot's response is that it does not
completely follow the instructions. The instructions say that the
program should ``display those numbers with their sum'', but Copilot
only outputs the sum. The program is able to properly add the two
numbers together, but does not display them properly.

One other thing to note is how there is no indication that the program
was written by an AI. While a teacher could count the number of new
lines between the different sections of the program, this wouldn't be a
perfect indicator of whether a program was written by AI or not. In
addition, things like this are simple for a student to fake.

\hypertarget{conclusion}{%
\section{Conclusion}\label{conclusion}}

From this and from other past experimentation, Copilot is imperfect.
Using it isn't plagiarism, but using it without informing your professor
could be considered academic dishonesty. The initial testing shows that,
while the AI is imperfect, it is still able to produce very nearly
perfect submissions.

In his article \emph{Your Wish Is My CMD}, Neil Savage points out that
very little code on sources like Github is labeled with its intention
\textbf{FOOTNOTE}. Even if it can generate runnable code, from this
experiment, Copilot is not always able to generate code that perfectly
follows the given instructions. It can generally follow the
instructions, but it often needs some human correction to complete a
task properly. This lines up with my experience testing Copilot. Copilot
can generate functions, classes, and other code incredibly quickly and
accurately; the problem comes when Copilot tries to generate code based
off of a very specific prompt. It sometimes gets it right, but more
often than not, it'll end up generating a related response which does
part of the task provided, but not all of it.

Submissions generated via Copilot and other tools would be nearly
indistinguishable from normal submissions, especially in a large
classroom. In addition, from my other (undocumented) experience working
with Copilot, a simple knowledge of a language and some simple
manipulation of Copilot is more than enough to make the bot generate
perfect, working code. While Copilot may not be a perfect student to
cheat with yet, given a bit of time or give it to a student with a
cursory understanding of programming, and Copilot will be a formidable
weapon used against professors.

\hypertarget{further-study}{%
\section{Further Study}\label{further-study}}

In the next year or so, I am planning a much larger experiment based on
Copilot's capabilities. While I am still hashing out an exact
methodology on
\href{https://github.com/2br-2b/Github-Copilot-Project/blob/master/Methodology.md}{Github},
the result will end with having professors grade one extra assignment
each. The submission will be either generated by AI or taken from the
internet. As it will be a blind study, professors will not be told the
source of their assignment and will be asked to grade the code along
with student submissions. They will then take the grade, compare it to
other students' grades, and see if Copilot does better, worse, or
average in their class. I am still working on the exact details, but if
you are interested in helping with this process, please contact me via
the email listed on \href{https://github.com/2br-2b}{my Github profile}
and let me know! Thank you for your interest in my project!

\end{document}
