\documentclass{article}
\usepackage[utf8]{inputenc}
\usepackage{blindtext}
\usepackage{hyperref}

\title{Junior Seminar Paper}
\author{John Wuller}
\date{October 2021}

\begin{document}

\maketitle


\section{Abstract/Intro}

AI has been developing rapidly throughout the recent years. The release of GPT-3 has shown a remarkable development in AI text generation as compared to \textbf{the past}. In 2021, Microsoft introduced their new tool Github Copilot, an AI trained on publicly available code from Github, to generate code suggestions for developers. As this and similar tools become more and more prevalent, teachers and those in the academic field will need to prepare themselves as students begin to take advantage of these tools. When professors give their students assignments, if they are dishonest, students can already search the internet and plagiarize code from sources like StackOverflow. The problem with this approach is that the code is often copied verbatim, which means that professors can create their own programs to determine if a student's code is found online. There are other methods students can use to avoid coding themselves, but Github Copilot and similar tools will end up being more viable options for them.

This brings up an important question: is using Github Copilot actually cheating? My goal in this paper is to answer this question from my perspective as a student and to \_\_\_. We will first look at the potential legal issues with using Github Copilot, then look at the ways students can already cheat \_\_\_.

\section{Legal issues}

The first question that must be addressed is that of legality. Does using Github Copilot violate any copyright or other laws?

Specifically, let's look at plagiarism. If a student were to copy a philosophy paper from online, for example, this would be an example of academic dishonesty, as the student would be claiming someone else's work to be their own. Even if they were to ``cite their sources'' and put a quote mark at the beginning and end of the paper and indicate that the entire paper was just one long quote from the internet, most professors would reject this paper and not give the students credit for it. On the other hand, if the source isn't cited, there could be legal issues related to stealing another's intellectual property.

Of course, there's a whole other question of whether Github Copilot itself is in violation of legal rights from being trained on code under open-source licenses, but that is a subject for another paper.

In the same way as lifting entire paragraphs from the internet is disallowed, if a student were to generate a majority of their submission via Github Copilot, a teacher would most likely reject the assignment. After all, the student didn't do the work and didn't show that they have learned whatever was intended by the professor, so they don't deserve a good grade.

However, while teachers might not accept assignments generated by AI, they first would need to know that the assignment was generated by an AI. The problem is that, from a legal/copyright standpoint, students do not need to give credit to Github Copilot if they use it.

According to Github Copilot's website, ``GitHub Copilot is a tool, like a compiler or a pen. The suggestions GitHub Copilot generates, and the code you write with its help, belong to you, and you are responsible for it.'' Copilot's creators claim that Copilot, like an IDE, is merely a tool and that you have complete rights to all the code it generates. If I write code using Visual Studio Code, that doesn't mean that Microsoft has full rights to my code! The code I write is still \emph{my} code, and I hold all the rights to that code. It's just like with paper and pencils - if I use a Pilot pen sketch out an idea on a napkin, I still maintain full legal rights to the idea. Neither the pen manufacturer nor the napkin manufacturer can take your idea - it's \emph{your} idea. You simply used their tools to communicate the idea.

In addition, unlike papers found online, you do not need to credit Github Copilot in your code. As they say on their website, ``the code you create with GitHub Copilot's help belongs to you. While every friendly robot likes the occasional word of thanks, you are in no way obligated to credit GitHub Copilot. Just like with a compiler, the output of your use of GitHub Copilot belongs to you.''

Thus, from a legal/copyright standpoint code generated by Github Copilot would be fair use for students. If a student were to use Github Copilot without mentioning it to a professor, that student wouldn't be able to be legally able to be prosecuted.

This, of course, assumes that Copilot acts like a compiler or an IDE. Let's compare how Copilot acts as compared to other tools which can be integrated into an IDE.

\section{Different levels of autocomplete}

Over time, programmers have worked to make their lives easier. They write tools which help them write programs easily, tools to autocomplete their code. While the levels are much more granular, we will limit our discussion to a few levels of tools which can be used to help developers.

The first level is to use no suggestions at all. This would be like using a app like Notepad or TextEdit to write code. There would be absolutely no help whatsoever for the developer. While this is certainly the hardest way to write code, it would also force the students to remember every rule in their programming language, from closing parenthesis to ending each line with a semicolon.

The next level of assistance would be to use a simple editor with a few more tools. This editor would include highlighting, parenthesis matching, auto indent, and a few other tools to help users on their way. Apps like Vim and Gedit fall into this category. They make it easier to find mistakes and to fix them, but you still need to remember every rule of your language. It helps, but you still need to know how to write Java programs in a given language.

The next level would be to use an editor with syntax highlighting and intelligent code completion. This would give you suggestions when you're typing for function and class names. This encompasses most standard IDEs used in programming classes, like (default) Visual Studio Code with IntelliSense. You still need to know how to code, but you don't need to remember the exact name of every class or method. Most people using these tools will already know what function they want to use when they start typing, but it could be used by some students if they don't always remember how to get a random integer between two other numbers in python.

The next higher level is to use an editor with snippets. Snippets are standard coding patterns that the editor will automatically complete for you if you ask for it. For example, if you type \texttt{functionname}, it can autocomplete the header \texttt{public\ static\ void\ functionname()\{\}}. If you have private variables in a Java class, they can be used to automatically generate getter and setter methods for it. Snippets are used to remove the monotony of rewriting the same code over and over again, but they still require you to know most of the language.

Finally, Github copilot is on a whole other level. With copilot, you need to know next to nothing about how a language works in order to complete an assignment using it. If the user is lucky, they won't even need to know what their code is doing if their generated code is good enough. As long as it works, it works, so some students would have no problem plugging an assignment into Copilot and generating and testing 10 different responses to determine if any of them pass all the test cases.

Ultimately, then, the main difference between Copilot and the other tools is that Copilot doesn't require you to know the language you're using while all of the other tools do require you to know the language you're being taught. Ultimately, then, from a learning perspective, the other tools can be allowed as they force you to learn the language, but only Copilot can be used by students to pretend that they know a given language.

\section{Current methods of cheating}

Github Copilot, of course, will only be a good tool for cheaters if it's better then the alternatives. It would be wise to compare cheating using Github Copilot to cheating using other tools.

The first (and perhaps most archaic way) would be to copy a friend's code. Perhaps one of the best-known methods of cheating, it requires a less-competent student to find a more competent student and steal their work. This is differentiated from asking another student to help you because you will not end up doing any of the work - you will simply submit the exact same assignment as another. Often, not many changes will be made to the assignment (sometimes, only the student's name on the work will change) before it is submitted. The benefit with this is that you will likely trust the source of the code to be accurate. However, if professors saw that two students submit the exact same assignment with the exact same variable names, syntax, etc., it would be obvious that the students were cheating. In addition, if the better student makes a mistake, it will also be copied over into the other student's work.

Another source of cheating would be to ask or to pay an expert to do your work for you. In this case, you choose a non-student (perhaps someone who has already taken the class or someone from Fiver) and have them do the assignment and give you the work. The benefit is that your work is unique and no one else in the class will have the same submission as you. The downside is that, if you're asked any questions about your work, you won't be able to answer them.

Closely related to this would be to search for your assignment on the internet. Oftentimes, different professors may assign similar (if not the same) assignments. If you search for your assignment on the internet, you can often find a lot of other people's work. Even if your specific question hasn't been answered, you can post the question yourself on Reddit or some other internet forum and ask for answers. The problem becomes that, since these answers are publicly available, plagiarism bots can scan for assignments lifted from the internet and flag them for teachers to review.

The question then becomes how Github Copilot stacks up to these other methods. Is its AI-generated code as good (or even better than) assignments stolen from these sources? 

\end{document}
